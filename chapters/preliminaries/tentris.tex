\section{Triple Stores and Tentris }
\label{sec:preliminaries:tentris}

\setlength{\parindent}{4ex}Triple stores are special kind of data management systems designed to store RDF triple data. An RDF triple is composed of three elements in order: \textit{a subject}, \textit{a predicate} and \textit{an object}. Each of these elements is called an \textit{RDF Term}  (RT). 
In turn, each RDF term value can either be \textit{an Internationalized Resource Identifier} (IRI), \textit{a blank node}, or \textit{a literal} \cite{RDF11}. 
For example, \verb|<http://example.com/Bob http://example.com/name "Bob">| is an RDF triple instance. \\

Generally, triple stores provide a standard interface to enable performing queries and other semantic operations on the stored RDF triples through a query language such as SPARQL \cite{Hitzler}. 
Triple stores are considered central elements in the “Storage and Querying” phase of the linked data life cycle \cite{LOD}.  \\

As there is no standard design guideline for triple stores, different implementations of triple stores co-exist. Each subgroup of these implementations utilizes a category of underlying data structures as well as corresponding algorithms that govern the behaviour. 
In production, triple stores are used to store up to billions of RDF triples. 
To that extent, quality factors like efficiency and scalability are considered first-class citizens during the construction of triple stores. 
And the selection of internal data structures and the behaviour definitions greatly influence the overall system efficiency. \\

Fuseki, Blazegraph, Virtuoso and RDF-3X are popular implementations of Triple stores. 
One of the common design characteristics is that they all utilize B+ trees for storing the indices. 
Other categories of triple stores use 3D Boolean tensors to store and process RDF data. In such systems, each tensor dimension is mapped to a triple data aspect, i.e. subject, predicate or object. Examples of tensor-based triple stores include systems like TensorRDF and BitMat. \\

Mention the \verb|.nt| file.

\subsection{Mapping RDF to Tensor}

\subsection{Hypertrie}