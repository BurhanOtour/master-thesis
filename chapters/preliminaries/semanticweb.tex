\section{Semantic Web}
\label{sec:preliminaries:semanticweb}

Today World Wide Web (WWW) holds a tremendous amount of data of different types (videos,
images, text, geolocation data, books, publications, etc.). Such data comes from various sources
(web applications, warehouse systems, GPS devices, smartphones, ATMs, etc.). However, such
data are still processed passively by computer systems as there is no way to understand its meaning or context.
Tim Berners-Lee coined the term \textit{Semantic Web} for a Web of Data (or Data Web) \cite{LeeWeavingTheWeb} that can be processed by machines.   
The key technologies of Semantic Web are published by the World Wide Web Consortium (W3C). 
The Resource Description Framework (RDF), a standard for representing data in the Semantic Web, and SPARQL, a query language for RDF data, are introduced in this section.

\subsection{Resource Description Framework}
\label{sec:rdf}
The Resource Description Framework (RDF) \cite{rdfonline} is part of the W3C standard to define the web of data. Regardless of the data nature held on the web (blog posts, images, publications, newspaper articles,  list of invoices, etc.), RDF identifies them uniformly as resources. In the standard, each resource is attached to a unique Internationalized Resource Identifier (IRI). IRI is a standard defined by the Internet Engineering Task Force in RFC 3987 \cite{rfc3987}. Literals are another sort of resources. A literal comprises a hardcoded value represented as a string; (“Martin,” “true”, “12.3”) are examples of literals. The third resource type is called a blank node. Blank nodes represent anonymous resources and always have local scope where they can be assigned a unique identifier. All definitions in this section are taken from “RDF 1.1 Concepts and Abstract Syntax” \cite{RDF11}. \\

\begin{definition}[RDF Terms]
Let $I$ be the set of \textit{IRIs}, $L$ be the set of \textit{literals} and be $B$ the set of
\textit{blank nodes}. Further $I$, $L$ and $B$ are finite and pair-wise disjoint. Then the set $RT = I \cup L \cup B$ is called the set of \textit{RDF terms}. 
\end{definition}

\subsubsection{RDF Triple, RDF Graph}
\label{sec:rdf_triples}
On an abstract level, an RDF triple can be seen as a statement that describes the relationship between two resources. An RDF triple can also serve to describe the property of a resource. RDF triple is represented by a sentence composed of three elements in order: \textit{a subject}, \textit{a predicate} and \textit{an object}.
The subject is an RDF resource. The subject has a property defined by the predicate and a value for that property set by the object.
RDF restricts which RDF terms can be used for subject, predicate, and object:

\begin{definition}[RDF Triple, RDF Graph]
An RDF Triple is a tuple $(s, p, o) \in  $ where $s$ is called subject, $p$ is called predicate and $o$ is called object. \\
A set of of RDF Triples is called RDF graph.
\end{definition}

\begin{example}
	An example of an RDF graph is given in Table \ref{tab:rdf_graph_example}. 
	The data in the graph presents a list of pioneers, along with their occupations and spouses. Table \ref{tab:rdf_graph_abbr} defines the abbreviations used in the example. \\
\end{example}

We use a particular query language to retrieve data from an RDF graph called SPARQL. SPARQL \cite{Hitzler} is a recursive acronym for “SPARQL Protocol And RDF Query Language”. Listing \ref{lst:sparqlexample} shows a simple example of a SPARQL query against the RDF graph in table \ref{tab:rdf_graph_example}. The SPARQL language semantic is built on top of a formal language called SPARQL-algebra \cite{Hitzler}. A core aspect of SRARQL-algebra is a sub-language called SPARQL-BGP (Basic Graph Pattern). \\

\begin{definition}[Triple Pattern, Basic Graph Pattern]
Let $V$ be an infinite set of variables disjoint to the set of RDF terms $RT$. The triple $(s, p, o) \in ( I \cup B \cup V) \times (I \cup V) \times (RT \cup V)$ is called \textit{triple pattern (TP)}. A set of triple patterns is called \textit{basic graph pattern (BGP)}.
\end{definition}

\verb|<?person ex:occupation ex:Doctor>| is a triple pattern with a variable \verb|?person| placed in the subject position. Applying a triple pattern $tp$ to an RDF graph $g$ (denoted by $P(g)$) finds all RDF triples where the predicate equals \verb|ex:occupation| and object equals \verb|ex:Doctor|. For each triple's occurrence, a mapping from \verb|?person| to the corresponding subject of the triple is created. As a result, we end up with a set of mappings to variable \verb|?person|. We call this set, the \textit{solution mappings} denoted with $\Omega$.  Two solution mappings are said to be \textit{compatible} when values assigned to the common variables are identical. Applying BGP on $g$ is done by finding compatible solution mappings resulted from its triple patterns. \\

In SPARQL, raw \verb|SELECT| operation is primarily based on evaluating basic graph patterns using the \verb|join| operator defined in SPARQL algebra in the context of finding compatible solution mappings \cite{Hitzler}.

\begin{table}[h]
	\centering
	\begin{tabular}{lll}
		\textbf{subject} & \textbf{predicate} & \textbf{object} \\ \hline
		ex:PresidentOfUS          &        rdf:type            &        madsrdf:Occupation       \\
		ex:Professor          &        rdf:type            &        madsrdf:Occupation       \\
		ex:PoliticalParty			&          rdf:type          &          v:Organization     \\
		ex:RepublicanParty			&          rdf:type          &          ex:PoliticalParty     \\
		ex:DemocraticParty			&          rdf:type          &          ex:PoliticalParty     \\
		ex:AndrewNg          &        madsrdf:occupation           &      ex:Professor         \\
		ex:BarakObama          &        madsrdf:occupation           &       ex:PresidentOfUS         \\
		ex:GeraldFord          &        madsrdf:occupation           &       ex:PresidentOfUS         \\
		ex:BarakObama          &        ex:party           &       ex:DemocraticParty	         \\
		ex:GeraldFord          &        ex:party           &       ex:RepublicanParty         \\
		ex:BarakObama          &        foaf:spouse           &       ex:DemocraticParty	         \\
		ex:GeraldFord          &        foaf:spouse           &       ex:MichelleObama         \\
		ex:AndrewNg          &        foaf:spouse           &       ex:CarolEReiley         \\
	\end{tabular}
	\caption{An example of an RDF graph.}
	\label{tab:rdf_graph_example}
\end{table}

\begin{table}[h]
	\centering
	\begin{tabular}{ll}
		\textbf{abbr.} & \textbf{IRI}  \\ \hline
		rdf:         &        http://www.w3.org/1999/02/22-rdf-syntax-ns       \\
		madsrdf:          &        	http://www.loc.gov/mads/rdf/v1                 \\
		ex:			&          https://www.example.com            \\
		v:			&          https://www.w3.org/TR/vcard-rdf/          \\
		foaf:			&         http://xmlns.com/foaf/0.1/           \\
	\end{tabular}
	\caption{The list of abbreviations used in table \ref{tab:rdf_graph_example}.}
	\label{tab:rdf_graph_abbr}
\end{table}


\begin{lstlisting}[label={lst:sparqlexample},caption={A SPARQL query that returns the spouse of each US president whose party is the democratic.}]
SELECT ?spouse
WHERE { 
	?president madsrdf:occupation ex:PresidentOfUS .
	?president ex:party ex:DemocraticParty .
	?president foaf:spouse ?spouse }
\end{lstlisting}
 
 \subsubsection{Triple Stores}
 \textit{Triple Stores} are a special kind of data management system designed to store RDF triple data. 
 It can store one or more RDF graphs. 
 Generally, triple stores provide a standard interface to enable performing queries and other semantic operations on the stored RDF triples using a query language such as SPARQL (Sec. \ref{sec:rdf_triples}).
 Triple stores usually expose HTTP-based SPARQL endpoints for querying enablement. 
 A triple store client can send a SPARQL query embedded in an HTTP request to the endpoint.
 The store evaluates the query against the RDF data it holds and returns the results to the client in the body of an HTTP response in a format conforms to the system configuration.
 
\clearpage