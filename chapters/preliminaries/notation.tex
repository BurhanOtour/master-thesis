\section{Notation and Convention}
\label{sec:notation}

For a function $f$, the domain and co-domain of $f$ are denoted by dom($f$) and codom($f$), respectively. The set of natural numbers $\mathbb{N}$ includes zero in this work. Further $\mathbb{N}_{n}$ with $n \in \mathbb{N}$ is equal to $\{0, 1, ..., n -1\}$. Let $\mathbb{B}$ be the set of boolean values; i.e. $\mathbb{B}=\{true, false\}$. We map $true$ to 1 and $false$ to 0. We use angle brackets $<...>$ to define a tuple $t$ which represents a sequence with fixed order for its elements. The entries $<t_0, t_1, ... , t_{n-1}>$ of a tuple $t$ with length $n$ can be accessed using the square bracket notation (subscript) after the tuple symbol. 
For example, $t[i]=t_i$ is the tuple $t$ entry at position $i$. 
Entries of a tuple $t$ are zero-indexed. 
The domain of a tuple $t$, denoted by dom($t$) = $<0,1,...,n-1>$, is a tuple of $t$ entries' positions.

