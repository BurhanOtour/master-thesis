\chapter{Introduction}
\label{ch:introduction}
RDF triple stores are considered central elements in the "Storage and Querying" phase of the linked data life cycle \cite{axel}.
Tentris is a triple store that realizes the concept of storing the entire RDF graph in a sparse order-3 boolean tensor using a novel trie-based data structure, dubbed hypertrie.
Among others, Tentris performance surpasses other well-known open-source and commercial triple stores \cite{tentris2020} for single and multiple simultaneous SPARQL clients. Despite its high-performance characteristic, the Tentris system still suffers from excessive memory utilization when it tends to store large RDF graphs. hypertrie, on average, requires four times the space needed to store the RDF graph in a list. In my master thesis, I contribute to the development of Tentris by researching and implementing an approach to practically reduce the overall system memory footprint using a known technique in tree-based data structures called path compression. My effort is reflected mainly in extending the hypertrie code base to program the space reduction approach. The space reduction implementation is preserved in the GitHub branch: \verb|https://github.com/dice-group/hypertrie/tree/path_compression|. Key contributions of the thesis are:
\begin{itemize}
	\item Design an approach to reduce the utilized memory space in hypertrie.
	\item Expanding the hypertrie code base to add an implementation to the approach using modern C++.
	\item Define and implement test cases (unit and integration tests) to check my implementation's conformance to the main hypertrie contracts defined in its interfaces.
	\item Evaluate the approach by executing bulk loading and benchmark experiments against real RDF data sets.
\end{itemize}


\paragraph{Outline} I first introduce the main concepts upon which this work is built in chapter 2. I present the idea of tagged pointers and provide a quick overview of the semantic web and introduce Tentris and the logic behind it in a bit of detail. Chapter 3 lists some contributions from the scientific community toward developing space-efficient in-memory data structures to store sequences. Afterwards, the path compression approach to minimize the memory footprint of hypertrie is introduced in chapter 4. I discuss the approach's central concept and the refactoring work required on hypertrie on both the node design and the behavior levels. Evaluating the implementation of the space reduction approach by executing bulk loading and benchmarks experiments are shown in chapter 5. Chapter 6 concludes with a final discussion and future work. 