\chapter{Makefiles}
\label{ch:makefiles}

This \acs{template} template comes with two Makefiles to make your life easier.
One of the files is located inside the template's main folder, the other one is
located in the figures sub-folder.
The main Makefile can be used to compile the thesis's source files into a PDF
file, while the figures directory's Makefile is used for creating graphics out
of certain types of raw data.
In this chapter, we discuss what both files do.

You can compile your thesis without these Makefiles.
However, the Makefiles make sure the compilation process happens in the right
order.
For a simple compilation, type ``make'' into your console and hit Enter.
This starts the simple compilation process (calling pdflatex twice).
Type ``make complete'' and hit Enter, in order to compile the source files,
and include compiled images and bibliographic references.
This latter option may pose problems in very specific circumstances.
Therefore we provide a section on troubleshooting next.



\section{Makefile Troubleshooting}
\label{sec:makefiles:trouble}
The Makefiles introduce certain dependencies, \ie{} applications they need to
properly function.
However, they will typically not pose a problem.
The issues that may occur are:
\begin{enumerate}
	\item your thesis's bibliography file is empty, or
	\item dependencies are not installed, but relevant files exist.
\end{enumerate}

\paragraph{Empty bibliography file.}
The compilation process may fail if the \mbox{bibliography.bib} file does not
contain a reference. In this case, either add a reference, or change Line~8 of
the main Makefile from  \verb+complete: images lit short+ to 
\verb+complete: images short+, or comment out Lines~12 and 13 of the main
Makefile by prepending them with a hashtag (\#).
The first option is to be preferred if you already know at least one
bibliographic reference that you will use, the second option is reasonable if
you will not have any bibliographic reference (very unlikely), and the third
option is more or less a quick and dirty hack.

\paragraph{Missing dependencies.}
The compilation process may also fail, if you have \mbox{.svg} files in your
figures directory, but \emph{inkscape} is not installed, or you have
\mbox{.tex} files in the figures directory but either \emph{pdflatex} or
\emph{pdfcrop} is not installed.
In either case, you can install the relevant dependencies, or remove the
problematic files, or comment out the problematic lines of the figures
directory's Makefile by prepending them with a hashtag (\#).
Which lines to comment out depend on the file type that causes the problems;
see Section~\ref{sec:makefiles:figures} for details.




\section{Main Makefile}
\label{sec:makefiles:main}
The main Makefile is designed to compile the \LaTeX{} source files of your
thesis into the final PDF file.
The Makefile provides two main recipes for obtaining the PDF, a short and a
long version.
The short version is invoked via ``make'' or ``make short,'' the long
version is called via ``make complete.''
 
We now have a look at the Makefile itself.
The first line stores the file name (without the file extension) of your
thesis's main \LaTeX{} file in variable \texttt{THESIS}
Typically, this is \mbox{thesis.tex}, and THESIS store the value ``thesis,''
accordingly.

Lines~4 to 6 are the recipe for the short compilation process, they simply call
pdflatex twice on your thesis's main source file.

Line~8 contains the recipe for the long compilation process. 
However, rather than containing the process itself, it depends on three other
recipes that are called automatically: ``images,'' ``lit'' and ``short.''
Hence (see below and above) the long compilation process created image files
from raw data in the figures directory, processes bibliographic references and
finally puts the PDF together.

Lines~10 to 12 process bibliographic references.
First, \emph{pdflatex} is invoked on your thesis's main file in order to
collect the bibliographic references that you actually use, and then 
\emph{bibtex} for processing bibliographic references is called.

Lines~14 to 15 compile raw image data from the figures directory.
The do so by changing into the directory and executing the directory's
Makefile; for details, see the next \hyperref[sec:makefiles:figures]{section}.



\section{figures's Makefile}
\label{sec:makefiles:figures}
The figures sub-folder's Makefile is designed to process raw data and transform
them into useful images.
It handles vector graphics in the \mbox{.svg} format, as well as graphics
described via tikz\footnote{\TeX{} Ist Kein Zeichenprogramm, a \LaTeX{}
package.} in \mbox{.tex} files.
The reasoning behind putting tikz graphics in separate files is that compiling 
the code into graphics is somewhat time consuming, so recompiling graphics that
do not change should be avoided.

We now go through the Makefile and describe its purposes.
The first line collects all \mbox{.svg} files in the ``figures'' directory, and
stores their file names in variable \texttt{SVG\_FILES}.
The second line creates a variable \texttt{INKSCAPE\_FILES} and initializes it
with all the file names stored in \texttt{SVG\_FILES} with the \mbox{.svg} file
extension replaced by \mbox{.pdf\_tex}.
Similarly, Lines~4 and 5 collect the \mbox{.tex} files and corresponding
\mbox{.pdf} files in variables \texttt{TEX\_FILES} and \texttt{TIKZ\_FILES}.

Line~7 says that, by default, we want to create the \mbox{.pdf\_tex} files from
\mbox{.svg} files, and \mbox{.pdf} files from \mbox{.tex} files.
The following lines say how this transformation process works.
However, the transformation is only applied to source files that have changed
recently, \ie{} if the target file does not exist yet, or source file's last
modified date is newer than the target file's last modified date.

Lines~9 and 10 are the recipe for creating \mbox{.pdf\_tex} files out of
\mbox{.svg} files.
For this, the application \emph{inkscape} is used.
The recipe calls inkscape in command line mode and asks it to export the source
file's drawing and text as two separate files.
The first output file is a \mbox{.pdf} and contains the drawing itself.
The second file is the desired \mbox{.pdf\_tex} that is a \LaTeX{} file in
disguise.\footnote{The \mbox{.pdf\_tex} file extension helps with avoiding
confusion with other \mbox{.tex} files.}
It contains \LaTeX{} code that puts embeds the \mbox{.pdf} file into your
thesis, but also contains the text from the \mbox{.svg} file and overlays the
\mbox{.pdf} graphic with the text.
As a result, the text uses the same font as your document, and all text from
the \mbox{.svg} file gets interpreted as \LaTeX{} code.
\emph{Note} that due to some bug in the inkscape export, you may observe
\LaTeX{} errors during compilation if your \mbox{.pdf} file contains more than
a single page, \ie{} if the \mbox{.svg} file was created with inkscape in the
first place, you must put all graphic elements onto the same layer before
exporting.

Lines~12 to 14 are the recipe for getting \mbox{.pdf} files from \mbox{.tex}
files.
For that, the source file is compiled using pdflatex, as expected.
Afterwards, the resulting \mbox{.pdf} file is cropped to the drawing area, with
some surrounding margins.
This adds the applications \emph{pdflatex} and \emph{pdfcrop} as dependencies.
